%----------------------------------------------------------------------
\section{Computational Drug Design} 
The landscape of drug discovery is being reshaped by computational drug design, an interdisciplinary
field that harnesses computational models, simulations, and data analysis to streamline the
identification and development of novel therapeutics \cite{schneider2020rethinking}. Computational
Drug Design encompasses a spectrum of techniques, from structure-based methods like molecular
docking \cite{kitchen2004docking} and virtual screening \cite{shoichet2004virtual} to ligand-based
approaches like quantitative structure-activity relationship (QSAR) modeling
\cite{cherkasov2014qsar} and machine learning algorithms \cite{gawehn2016dl}. \\

Recent advances in artificial intelligence, particularly deep learning, have propelled computational
drug design into a new era. Deep learning models, trained on massive datasets of molecular
structures and bioactivity data, can predict molecular properties, binding affinities, and even
generate novel drug-like molecules \cite{zeng2022deep} These AI-powered tools are not only
accelerating the traditionally time-consuming and costly drug discovery process but also expanding
the druggable chemical space and enabling the design of personalized medicines \cite{gawehn2016dl}

% TODO: add figure

%----------------------------------------------------------------------
\section{Generative Artificial Intelligence} 
The rise of Generative Artificial Intelligence (Generative AI) has ignited a wave of innovation across various domains,
from computer vision to natural language processing. Generative models like... TODO

%----------------------------------------------------------------------
\section{Peptide Therapeutics} 
Peptide therapeutics have garnered significant attention due to their unique advantages over
small-molecule drugs. Peptides, composed of short chains of amino acids, exhibit high target
specificity, minimal off-target effects, and reduced immunogenicity \cite{wang2022therapeutic}.
Their inherent biocompatibility and the ease of chemical modifications make them versatile tools for
therapeutic interventions. Peptides often contain a mix of hydrophilic and hydrophobic amino acids,
which can affect their solubility and interaction with cell membranes \cite{madani2011mech}. This
can impact their ability to pass through cell membranes and work effectively inside cells. Peptides
with a slightly higher percentage of hydrophobic residues may have increased cell permeability
\cite{madani2011mech} and be more effective at interacting with cell membranes, as hydrophobic
residues can interact very well with the hydrophobic regions of lipid bi-layers, enhancing the
transit of the peptide across cell membranes \cite{madani2011mech}. \\

Peptide-based drugs have already made a substantial impact in the treatment of various diseases. For
instance, insulin, a peptide hormone, has transformed diabetes management while peptide-based
antibiotics have effectively treated bacterial infections \cite{cesar2023adv}. The development of
peptide-based vaccines against infectious diseases and cancer is another exciting frontier in
peptide therapeutics \cite{purcell2007vaccine}. \\

Despite their promise, peptide therapeutics face challenges like limited oral bioavailability and
susceptibility to proteolytic degradation \cite{muttenthaler2021trends}. However, recent advances in
peptide engineering, such as cyclization \cite{hayes2021cycl}, incorporation of non-natural amino
acids \cite{li2022nonnat}, and the development of novel delivery systems, are addressing these
limitations and expanding the therapeutic potential of peptides.

%----------------------------------------------------------------------
\section{Malaria} 
Malaria, predominantly transmitted through bites of infected female Anopheles mosquitoes, remains a
significant global health burden, particularly in tropical and subtropical regions
\cite{duguma2022ethiopia, ghosh2021malaria}. Its severity is underscored by its lethality,
especially in young children and pregnant women. In 2019, the World Health Organization (WHO)
reported an estimated 229 million malaria cases and 409,000 deaths, highlighting the urgent need for
effective treatment alternatives \cite{jain2022pregnancy}. Around 6.3 million cases were reported
from the Southeast Asia region, majority of cases were present in India
\cite{cristina2020pregnancy}. \\

The \textit{Plasmodium falciparum} parasite, the most lethal species causing malaria, has developed
resistance to multiple antimalarial drugs, highlighting the urgent need for novel therapeutic
strategies \cite{fairhurst2012artemisinin}. This resistance hampers treatment
efficacy, potentially leading to prolonged illness, increased healthcare costs, and elevated
mortality risks. Beyond immediate mortality, malaria can have lasting detrimental effects on
individuals, even in non-fatal cases. Recurrent infections can contribute to anemia, cognitive
decline (particularly in children), and other complications, ultimately diminishing quality of life
and economic productivity \cite{shukla2022super}.
\\

Peptide therapeutics offer a promising avenue for combating malaria. Peptides can target specific
parasite proteins essential for survival and replication, potentially overcoming drug resistance
mechanisms \cite{li2009pfemp1}. Additionally, the lower likelihood of resistance development against
peptides compared to small molecules makes them attractive candidates for antimalarial drug
development. Recent efforts have focused on designing peptide inhibitors against key parasite
proteins like \textit{Plasmodium falciparum} Erythrocyte Membrane Protein 1 (PfEMP1), which plays a
crucial role in parasite sequestration and immune evasion \cite{jensen2020pfemp1}. Computational methods, particularly
generative AI approaches, have proven invaluable in accelerating the discovery of novel
peptide-based antimalarials by efficiently exploring the vast peptide sequence space and identifying
promising candidates with high affinity and specificity for critical parasite targets
\cite{chang2022rational}.
