Our novel hybrid deep learning approach, HYDRA, presents a significant advancement in target-aware peptide binder design. It goes beyond existing data-driven peptide design approaches by introducing a novel hybrid model that leverages a denoising diffusion model in conjunction with a binding affinity maximization algorithm. This approach utilizes a 3D-structure-specific representation of the protein binding site and the peptide, allowing for the generation of diverse, high-quality peptides tailored to specific binding locations. Due to the model's optimization of the binding affinity of protein-peptide complexes, it aims to generate peptides with enhanced stability and longer half-lives, potentially prolonging their therapeutic effect. HYDRA is also observed to generate peptides with a balanced ratio of hydrophilic and hydrophobic residues, enhancing their membrane contact and facilitating efficient cellular uptake.  \\

To showcase the model's capabilities, we successfully used it to design \textit{de novo} peptide binders for proteins expressed by PfEMP1 genes, key contributors to antigenic variation in the malaria parasite. By analyzing \textit{Plasmodium falciparum} parasite single-cell transcriptome data, we identified highly expressed PfEMP1 genes and subsequently pinpointed strong and medium binding sites within their structures using CAVITY \cite{Yuan2013}. Following peptide generation using HYDRA, we selected a subset of peptides exhibiting superior characteristics for each protein, demonstrating the model's ability to generate highly stable, cell-permeable, and potentially effective drug candidates. \\

Overall, our findings demonstrate the remarkable potential of hybrid deep learning approaches like HYDRA in revolutionizing peptide drug discovery. This work paves the way for the development of novel therapeutic peptides with improved stability, efficacy, and targeted delivery, ultimately contributing to advancements in healthcare. In the future, it might be worth exploring the possibility of using fully differentiable local peptide structure optimization and binding affinity computation algorithms \cite{wang2023fully} in order to construct a faster, end-to-end deep learning framework for \textit{de novo} design of stable, therapeutic peptides. It is also suggested that the inclusion of some unnatural (non-proteinogenic) amino acids in the peptide would benefit its stability \cite{adhikari2021reprogramming}. This aspect can also be explored in future studies. From a practical point of view, we acknowledge the limitation concerning the efficacy of the designed peptides in the real experimental and, more importantly, in clinical setup. However, the study would definitely facilitate potential peptide design for further experimental investigations. Recent studies have already demonstrated that machine learning-based peptide design protocols have successfully synthesized effective peptides. Current peptide design paradigms are still in their infancy, and we believe many more such investigations would accelerate its growth towards maturity in the near future. 
