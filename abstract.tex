The field of therapeutic peptide design is ripe for transformation, fueled by the convergence of biotechnology and artificial intelligence. Peptides, short chains of amino acids, offer a promising avenue for targeted drug therapies due to their inherent advantages over small molecules, including specificity and reduced side effects. However, the development of peptide therapeutics has been hindered by their limited oral bioavailability and susceptibility to enzymatic degradation. Recent advancements in deep learning techniques have opened new possibilities for addressing these challenges through innovative peptide design strategies.

This thesis explores the development of a novel hybrid deep learning framework for de novo peptide design. By harnessing the power of diffusion models, known for their ability to learn complex data distributions, and integrating them with binding affinity maximization algorithms, we have created a system capable of generating peptide sequences optimized for specific target receptors. To demonstrate the applicability of this framework, we focus on designing therapeutic peptides targeting proteins expressed by Plasmodium falciparum Erythrocyte Membrane Protein 1 (PfEMP1) genes, key contributors to malaria pathogenesis.

Our results highlight the potential of this hybrid deep learning approach to revolutionize peptide drug discovery. By generating peptide candidates conditioned on the binding sites of target receptors, we offer a promising avenue for developing effective therapies for malaria and other diseases. This research underscores the transformative power of AI in peptide therapeutics, paving the way for a new era of precision medicine with enhanced efficacy and reduced toxicity.
